\documentclass{article}
\RequirePackage{amsmath}
\RequirePackage{bytefield}
\RequirePackage{graphicx}
\RequirePackage{newtxmath}
\RequirePackage{mathtools}
\RequirePackage{xspace}
\RequirePackage{url}
\RequirePackage{changepage}
\RequirePackage{enumitem}
\RequirePackage{tabularx}
\RequirePackage{hhline}
\RequirePackage[usestackEOL]{stackengine}
\RequirePackage{comment}
\RequirePackage{needspace}
\RequirePackage[nobottomtitles]{titlesec}
\RequirePackage[hang]{footmisc}
\RequirePackage{xstring}
\RequirePackage[unicode,bookmarksnumbered,bookmarksopen,pdfview=Fit]{hyperref}
\RequirePackage{cleveref}
\RequirePackage{nameref}

\RequirePackage[style=alphabetic,maxbibnames=99,dateabbrev=false,urldate=iso8601,backref=true,backrefstyle=none,backend=biber]{biblatex}
\addbibresource{btch.bib}

% Fonts
\RequirePackage{lmodern}
\RequirePackage{quattrocento}
\RequirePackage[bb=ams]{mathalfa}

% Quattrocento is beautiful but doesn't have an italic face. So we scale
% New Century Schoolbook italic to fit in with slanted Quattrocento and
% match its x height.
\renewcommand{\emph}[1]{\hspace{0.15em}{\fontfamily{pnc}\selectfont\scalebox{1.02}[0.999]{\textit{#1}}}\hspace{0.02em}}

% While we're at it, let's match the tt x height to Quattrocento as well.
\let\oldtexttt\texttt
\let\oldmathtt\mathtt
\renewcommand{\texttt}[1]{\scalebox{1.02}[1.07]{\oldtexttt{#1}}}
\renewcommand{\mathtt}[1]{\scalebox{1.02}[1.07]{$\oldmathtt{#1}$}}

\newcommand{\zsendmany}{\textbf{z\_sendmany} }

% bold but not extended
\newcommand{\textbnx}[1]{{\fontseries{b}\selectfont #1}}


\crefformat{footnote}{#2\footnotemark[#1]#3}

\DeclareLabelalphaTemplate{
  \labelelement{\field{citekey}}
}

\DefineBibliographyStrings{english}{
  page  = {page},
  pages = {pages},
  backrefpage = {\mbox{$\uparrow$ p\!}},
  backrefpages = {\mbox{$\uparrow$ p\!}}
}

\setlength{\oddsidemargin}{-0.25in}
\setlength{\textwidth}{7in}
\setlength{\topmargin}{-0.75in}
\setlength{\textheight}{9.2in}
\setlength{\parindent}{0ex}
\renewcommand{\arraystretch}{1.4}
\overfullrule=2cm

\setlength{\footnotemargin}{0.6em}
\setlength{\footnotesep}{2ex}
\addtolength{\skip\footins}{3ex}

\renewcommand{\bottomtitlespace}{8ex}

% Use rubber lengths between paragraphs to improve default pagination.
% https://tex.stackexchange.com/questions/17178/vertical-spacing-pagination-and-ideal-results
\setlength{\parskip}{1.5ex plus 1pt minus 1pt}

\setlist[enumerate]{before=\vspace{-1ex}}
\setlist[itemize]{itemsep=0.5ex,topsep=0.2ex,before=\vspace{-1ex},after=\vspace{1.5ex}}

\newlist{formulae}{itemize}{3}
\setlist[formulae]{itemsep=0.2ex,topsep=0ex,leftmargin=1.5em,label=,after=\vspace{1.5ex}}

\newcommand{\docversion}{Pre-Release Version}
\newcommand{\termbf}[1]{\textbf{#1}\xspace}
\newcommand{\Hushlist}{\termbf{HushList}}
\newcommand{\HushList}{\termbf{HushList}}
\newcommand{\Hushlists}{\termbf{HushLists}}
\newcommand{\HushLists}{\termbf{HushLists}}

\newcommand{\BTCH}{\termbf{BTCH}}
\newcommand{\ZAU}{\termbf{ZAU}}
\newcommand{\doctitle}{Bitcoin Hush (\BTCH) Cryptocoin Specification}
\newcommand{\leadauthor}{Duke Leto}
\newcommand{\coauthora}{\;jl777}

\newcommand{\keywords}{privacy coin, cryptocurrency, UTXOs, anonymity, freedom of speech, cryptographic protocols,\
electronic commerce and payment, financial privacy, proof of work, zero knowledge, zkSNARKs}

\hypersetup{
  pdfborderstyle={/S/U/W 0.7},
  pdfinfo={
    Title={\doctitle, \docversion},
    Author={\leadauthor, \coauthora},
    Keywords={\keywords}
  }
}

\makeatletter
\renewcommand*{\@fnsymbol}[1]{\ensuremath{\ifcase#1\or \dagger\or \ddagger\or
    \mathsection\or \mathparagraph\else\@ctrerr\fi}}
\makeatother

\renewcommand{\sectionautorefname}{\S\!}
\renewcommand{\subsectionautorefname}{\S\!}
\renewcommand{\subsubsectionautorefname}{\S\!}
\renewcommand{\subparagraphautorefname}{\S\!}
\newcommand{\crossref}[1]{\autoref{#1}\, \emph{`\nameref*{#1}\kern -0.05em'} on p.\,\pageref*{#1}}

\newcommand{\nstrut}[1]{\texorpdfstring{#1\rule[-.2\baselineskip]{0pt}{\baselineskip}}{#1}}
\newcommand{\nsection}[1]{\section{\nstrut{#1}}}
\newcommand{\nsubsection}[1]{\subsection{\nstrut{#1}}}
\newcommand{\nsubsubsection}[1]{\subsubsection{\nstrut{#1}}}

\newcommand{\introlist}{\needspace{15ex}}
\newcommand{\introsection}{\needspace{30ex}}

\mathchardef\mhyphen="2D

% http://tex.stackexchange.com/a/309445/78411
\DeclareFontFamily{U}{FdSymbolA}{}
\DeclareFontShape{U}{FdSymbolA}{m}{n}{
    <-> s*[.4] FdSymbolA-Regular
}{}
\DeclareSymbolFont{fdsymbol}{U}{FdSymbolA}{m}{n}
\DeclareMathSymbol{\smallcirc}{\mathord}{fdsymbol}{"60}

\makeatletter
\newcommand{\hollowcolon}{\mathpalette\hollow@colon\relax}
\newcommand{\hollow@colon}[2]{
  \mspace{0.7mu}
  \vbox{\hbox{$\m@th#1\smallcirc$}\nointerlineskip\kern.45ex \hbox{$\m@th#1\smallcirc$}\kern-.06ex}
  \mspace{1mu}
}
\makeatother
\newcommand{\typecolon}{\;\hollowcolon\;}

% We just want one ampersand symbol from boisik.
\DeclareSymbolFont{bskadd}{U}{bskma}{m}{n}
\DeclareFontFamily{U}{bskma}{\skewchar\font130 }
\DeclareFontShape{U}{bskma}{m}{n}{<->bskma10}{}
\DeclareMathSymbol{\binampersand}{\mathbin}{bskadd}{"EE}

\newcommand{\hairspace}{~\!}
\newcommand{\hparen}{\hphantom{(}}

\newcommand{\hfrac}[2]{\scalebox{0.8}{$\genfrac{}{}{0.5pt}{0}{#1}{#2}$}}


\RequirePackage[usenames,dvipsnames]{xcolor}
% https://en.wikibooks.org/wiki/LaTeX/Colors#The_68_standard_colors_known_to_dvips
\newcommand{\todo}[1]{{\color{Sepia}\sf{TODO: #1}}}

\newcommand{\changedcolor}{magenta}
\newcommand{\setchanged}{\color{\changedcolor}}
\newcommand{\changed}[1]{\texorpdfstring{{\setchanged{#1}}}{#1}}

% terminology

\newcommand{\term}[1]{\textsl{#1}\kern 0.05em\xspace}
\newcommand{\titleterm}[1]{#1}
\newcommand{\quotedterm}[1]{``~\!\!\term{#1}''}
\newcommand{\conformance}[1]{\textbnx{#1}\xspace}

\newcommand{\Zcash}{\termbf{Zcash}}
\newcommand{\Hush}{\termbf{Hush}}
\newcommand{\Zerocash}{\termbf{Zerocash}}
\newcommand{\Bitcoin}{\termbf{Bitcoin}}
\newcommand{\Komodo}{\termbf{Komodo}}
\newcommand{\CryptoNote}{\termbf{CryptoNote}}
\newcommand{\ZEC}{\termbf{ZEC}}
\newcommand{\ZER}{\termbf{ZER}}
\newcommand{\ZEN}{\termbf{ZEN}}
\newcommand{\ZCL}{\termbf{ZCL}}
\newcommand{\KMD}{\termbf{KMD}}
\newcommand{\VOT}{\termbf{VOT}}
\newcommand{\BTCP}{\termbf{BTCP}}
\newcommand{\ZGLD}{\termbf{ZGLD}}
\newcommand{\HUSH}{\termbf{HUSH}}
\newcommand{\zatoshi}{\term{zatoshi}}
\newcommand{\puposhi}{\term{puposhi}}
\newcommand{\zcashd}{\textsf{zcashd}\,}
\newcommand{\hushd}{\textsf{hushd}\,}

\newcommand{\MUST}{\conformance{MUST}}
\newcommand{\MUSTNOT}{\conformance{MUST NOT}}
\newcommand{\SHOULD}{\conformance{SHOULD}}
\newcommand{\SHOULDNOT}{\conformance{SHOULD NOT}}
\newcommand{\ALLCAPS}{\conformance{ALL CAPS}}

\newcommand{\note}{\term{note}}
\newcommand{\notes}{\term{notes}}
\newcommand{\Note}{\titleterm{Note}}
\newcommand{\Notes}{\titleterm{Notes}}
\newcommand{\dummy}{\term{dummy}}
\newcommand{\dummyNotes}{\term{dummy notes}}
\newcommand{\DummyNotes}{\titleterm{Dummy Notes}}
\newcommand{\commitmentScheme}{\term{commitment scheme}}
\newcommand{\commitmentTrapdoor}{\term{commitment trapdoor}}
\newcommand{\commitmentTrapdoors}{\term{commitment trapdoors}}
\newcommand{\trapdoor}{\term{trapdoor}}
\newcommand{\noteCommitment}{\term{note commitment}}
\newcommand{\noteCommitments}{\term{note commitments}}
\newcommand{\NoteCommitment}{\titleterm{Note Commitment}}
\newcommand{\NoteCommitments}{\titleterm{Note Commitments}}
\newcommand{\noteCommitmentTree}{\term{note commitment tree}}
\newcommand{\NoteCommitmentTree}{\titleterm{Note Commitment Tree}}
\newcommand{\noteTraceabilitySet}{\term{note traceability set}}
\newcommand{\noteTraceabilitySets}{\term{note traceability sets}}
\newcommand{\joinSplitDescription}{\term{JoinSplit description}}
\newcommand{\joinSplitDescriptions}{\term{JoinSplit descriptions}}
\newcommand{\JoinSplitDescriptions}{\titleterm{JoinSplit Descriptions}}
\newcommand{\sequenceOfJoinSplitDescriptions}{\changed{sequence of} \joinSplitDescription\changed{\term{s}}\xspace}
\newcommand{\joinSplitTransfer}{\term{JoinSplit transfer}}
\newcommand{\joinSplitTransfers}{\term{JoinSplit transfers}}
\newcommand{\JoinSplitTransfer}{\titleterm{JoinSplit Transfer}}
\newcommand{\JoinSplitTransfers}{\titleterm{JoinSplit Transfers}}
\newcommand{\joinSplitSignature}{\term{JoinSplit signature}}
\newcommand{\joinSplitSignatures}{\term{JoinSplit signatures}}
\newcommand{\joinSplitSigningKey}{\term{JoinSplit signing key}}
\newcommand{\joinSplitVerifyingKey}{\term{JoinSplit verifying key}}
\newcommand{\joinSplitStatement}{\term{JoinSplit statement}}
\newcommand{\joinSplitStatements}{\term{JoinSplit statements}}
\newcommand{\JoinSplitStatement}{\titleterm{JoinSplit Statement}}
\newcommand{\joinSplitProof}{\term{JoinSplit proof}}
\newcommand{\statement}{\term{statement}}
\newcommand{\zeroKnowledgeProof}{\term{zero-knowledge proof}}
\newcommand{\ZeroKnowledgeProofs}{\titleterm{Zero-Knowledge Proofs}}
\newcommand{\provingSystem}{\term{proving system}}
\newcommand{\zeroKnowledgeProvingSystem}{\term{zero-knowledge proving system}}
\newcommand{\ZeroKnowledgeProvingSystem}{\titleterm{Zero-Knowledge Proving System}}
\newcommand{\ppzkSNARK}{\term{preprocessing zk-SNARK}}
\newcommand{\provingKey}{\term{proving key}}
\newcommand{\zkProvingKeys}{\term{zero-knowledge proving keys}}
\newcommand{\verifyingKey}{\term{verifying key}}
\newcommand{\zkVerifyingKeys}{\term{zero-knowledge verifying keys}}
\newcommand{\joinSplitParameters}{\term{JoinSplit parameters}}
\newcommand{\JoinSplitParameters}{\titleterm{JoinSplit Parameters}}
\newcommand{\arithmeticCircuit}{\term{arithmetic circuit}}
\newcommand{\rankOneConstraintSystem}{\term{Rank 1 Constraint System}}
\newcommand{\primary}{\term{primary}}
\newcommand{\primaryInput}{\term{primary input}}
\newcommand{\primaryInputs}{\term{primary inputs}}
\newcommand{\auxiliaryInput}{\term{auxiliary input}}
\newcommand{\auxiliaryInputs}{\term{auxiliary inputs}}
\newcommand{\fullnode}{\term{full node}}
\newcommand{\fullnodes}{\term{full nodes}}
\newcommand{\anchor}{\term{anchor}}
\newcommand{\anchors}{\term{anchors}}
\newcommand{\UTXO}{\term{UTXO}}
\newcommand{\UTXOs}{\term{UTXOs}}
\newcommand{\block}{\term{block}}
\newcommand{\blocks}{\term{blocks}}
\newcommand{\header}{\term{header}}
\newcommand{\headers}{\term{headers}}
\newcommand{\blockHeader}{\term{block header}}
\newcommand{\blockHeaders}{\term{block headers}}
\newcommand{\Blockheader}{\term{Block header}}
\newcommand{\BlockHeader}{\titleterm{Block Header}}
\newcommand{\blockVersionNumber}{\term{block version number}}
\newcommand{\blockVersionNumbers}{\term{block version numbers}}
\newcommand{\Blockversions}{\term{Block versions}}
\newcommand{\blockTime}{\term{block time}}
\newcommand{\blockHeight}{\term{block height}}
\newcommand{\blockHeights}{\term{block heights}}
\newcommand{\genesisBlock}{\term{genesis block}}
\newcommand{\transaction}{\term{transaction}}
\newcommand{\transactions}{\term{transactions}}
\newcommand{\Transactions}{\titleterm{Transactions}}
\newcommand{\transactionFee}{\term{transaction fee}}
\newcommand{\transactionFees}{\term{transaction fees}}
\newcommand{\transactionVersionNumber}{\term{transaction version number}}
\newcommand{\transactionVersionNumbers}{\term{transaction version numbers}}
\newcommand{\Transactionversion}{\term{Transaction version}}
\newcommand{\coinbaseTransaction}{\term{coinbase transaction}}
\newcommand{\coinbaseTransactions}{\term{coinbase transactions}}
\newcommand{\CoinbaseTransactions}{\titleterm{Coinbase Transactions}}
\newcommand{\transparent}{\term{transparent}}
\newcommand{\xTransparent}{\term{Transparent}}
\newcommand{\Transparent}{\titleterm{Transparent}}
\newcommand{\transparentValuePool}{\term{transparent value pool}}
\newcommand{\deshielding}{\term{deshielding}}
\newcommand{\shielding}{\term{shielding}}
\newcommand{\shielded}{\term{shielded}}
\newcommand{\shieldedXTN}{\term{shielded} $ t \rightarrow z $ transaction}
\newcommand{\shieldedXTNs}{\term{shielded} $ t \rightarrow z $ transactions}
\newcommand{\shieldedNote}{\term{shielded note}}
\newcommand{\shieldedNotes}{\term{shielded notes}}
\newcommand{\xShielded}{\term{Shielded}}
\newcommand{\Shielded}{\titleterm{Shielded}}
\newcommand{\blockchain}{\term{block chain}}
\newcommand{\blockchains}{\term{block chains}}
\newcommand{\mempool}{\term{mempool}}
\newcommand{\treestate}{\term{treestate}}
\newcommand{\treestates}{\term{treestates}}
\newcommand{\nullifier}{\term{nullifier}}
\newcommand{\nullifiers}{\term{nullifiers}}
\newcommand{\xNullifiers}{\term{Nullifiers}}
\newcommand{\Nullifier}{\titleterm{Nullifier}}
\newcommand{\Nullifiers}{\titleterm{Nullifiers}}
\newcommand{\nullifierSet}{\term{nullifier set}}
\newcommand{\NullifierSet}{\titleterm{Nullifier Set}}
% Daira: This doesn't adequately distinguish between zk stuff and transparent stuff
\newcommand{\paymentAddress}{\term{payment address}}
\newcommand{\paymentAddresses}{\term{payment addresses}}
\newcommand{\viewingKey}{\term{viewing key}}
\newcommand{\viewingKeys}{\term{viewing keys}}
\newcommand{\spendingKey}{\term{spending key}}
\newcommand{\spendingKeys}{\term{spending keys}}
\newcommand{\payingKey}{\term{paying key}}
\newcommand{\transmissionKey}{\term{transmission key}}
\newcommand{\transmissionKeys}{\term{transmission keys}}
\newcommand{\keyTuple}{\term{key tuple}}
\newcommand{\notePlaintext}{\term{note plaintext}}
\newcommand{\notePlaintexts}{\term{note plaintexts}}
\newcommand{\NotePlaintexts}{\titleterm{Note Plaintexts}}
\newcommand{\notesCiphertext}{\term{transmitted notes ciphertext}}
\newcommand{\incrementalMerkleTree}{\term{incremental Merkle tree}}
\newcommand{\merkleRoot}{\term{root}}
\newcommand{\merkleNode}{\term{node}}
\newcommand{\merkleNodes}{\term{nodes}}
\newcommand{\merkleHash}{\term{hash value}}
\newcommand{\merkleHashes}{\term{hash values}}
\newcommand{\merkleLeafNode}{\term{leaf node}}
\newcommand{\merkleLeafNodes}{\term{leaf nodes}}
\newcommand{\merkleInternalNode}{\term{internal node}}
\newcommand{\merkleInternalNodes}{\term{internal nodes}}
\newcommand{\MerkleInternalNodes}{\term{Internal nodes}}
\newcommand{\merklePath}{\term{path}}
\newcommand{\merkleLayer}{\term{layer}}
\newcommand{\merkleLayers}{\term{layers}}
\newcommand{\merkleIndex}{\term{index}}
\newcommand{\merkleIndices}{\term{indices}}
\newcommand{\zkSNARK}{\term{zk-SNARK}}
\newcommand{\zkSNARKs}{\term{zk-SNARKs}}
\newcommand{\libsnark}{\term{libsnark}}
\newcommand{\memo}{\term{memo field}}
\newcommand{\memos}{\term{memo fields}}
\newcommand{\Memos}{\titleterm{Memo Fields}}
\newcommand{\keyAgreementScheme}{\term{key agreement scheme}}
\newcommand{\KeyAgreement}{\titleterm{Key Agreement}}
\newcommand{\keyDerivationFunction}{\term{Key Derivation Function}}
\newcommand{\KeyDerivation}{\titleterm{Key Derivation}}
\newcommand{\encryptionScheme}{\term{encryption scheme}}
\newcommand{\symmetricEncryptionScheme}{\term{authenticated one-time symmetric encryption scheme}}
\newcommand{\SymmetricEncryption}{\titleterm{Authenticated One-Time Symmetric Encryption}}
\newcommand{\signatureScheme}{\term{signature scheme}}
\newcommand{\pseudoRandomFunction}{\term{Pseudo Random Function}}
\newcommand{\pseudoRandomFunctions}{\term{Pseudo Random Functions}}
\newcommand{\PseudoRandomFunctions}{\titleterm{Pseudo Random Functions}}

% conventions
\newcommand{\bytes}[1]{\underline{\raisebox{-0.22ex}{}\smash{#1}}}
\newcommand{\zeros}[1]{[0]^{#1}}
\newcommand{\bit}{\mathbb{B}}
\newcommand{\Nat}{\mathbb{N}}
\newcommand{\PosInt}{\mathbb{N}^+}
\newcommand{\Rat}{\mathbb{Q}}
\newcommand{\typeexp}[2]{{#1}\vphantom{)}^{[{#2}]}}
\newcommand{\bitseq}[1]{\typeexp{\bit}{#1}}
\newcommand{\byteseqs}{\typeexp{\bit}{8\mult\Nat}}
\newcommand{\concatbits}{\mathsf{concat}_\bit}
\newcommand{\listcomp}[1]{[~{#1}~]}
\newcommand{\for}{\text{ for }}
\newcommand{\from}{\text{ from }}
\newcommand{\upto}{\text{ up to }}
\newcommand{\downto}{\text{ down to }}
\newcommand{\squash}{\!\!\!}
\newcommand{\caseif}{\squash\text{if }}
\newcommand{\caseotherwise}{\squash\text{otherwise}}
\newcommand{\sorted}{\mathsf{sorted}}
\newcommand{\length}{\mathsf{length}}
\newcommand{\mean}{\mathsf{mean}}
\newcommand{\median}{\mathsf{median}}
\newcommand{\clamp}[2]{\mathsf{clamp\,}_{#1}^{#2}}
\newcommand{\Lower}{\mathsf{lower}}
\newcommand{\Upper}{\mathsf{upper}}
\newcommand{\bitlength}{\mathsf{bitlength}}
\newcommand{\size}{\mathsf{size}}
\newcommand{\mantissa}{\mathsf{mantissa}}
\newcommand{\ToCompact}{\mathsf{ToCompact}}
\newcommand{\ToTarget}{\mathsf{ToTarget}}
\newcommand{\hexint}[1]{\mathbf{0x{#1}}}
\newcommand{\dontcare}{\kern -0.06em\raisebox{0.1ex}{\footnotesize{$\times$}}}
\newcommand{\ascii}[1]{\textbf{``\texttt{#1}"}}
\newcommand{\Justthebox}[2][-1.3ex]{\;\raisebox{#1}{\usebox{#2}}\;}
\newcommand{\hSigCRH}{\mathsf{hSigCRH}}
\newcommand{\hSigLength}{\mathsf{\ell_{hSig}}}
\newcommand{\hSigType}{\bitseq{\hSigLength}}
\newcommand{\EquihashGen}[1]{\mathsf{EquihashGen}_{#1}}
\newcommand{\CRH}{\mathsf{CRH}}
\newcommand{\CRHbox}[1]{\SHA\left(\Justthebox{#1}\right)}
\newcommand{\SHA}{\mathtt{SHA256Compress}}
\newcommand{\SHAName}{\term{SHA-256 compression}}
\newcommand{\FullHash}{\mathtt{SHA256}}
\newcommand{\FullHashName}{\mathsf{SHA\mhyphen256}}
\newcommand{\Blake}[1]{\mathsf{BLAKE2b\kern 0.05em\mhyphen{#1}}}
\newcommand{\BlakeGeneric}{\mathsf{BLAKE2b}}
\newcommand{\FullHashbox}[1]{\FullHash\left(\Justthebox{#1}\right)}
\newcommand{\setof}[1]{\{{#1}\}}
\newcommand{\range}[2]{\{{#1}\,..\,{#2}\}}
\newcommand{\minimum}{\mathsf{min}}
\newcommand{\maximum}{\mathsf{max}}
\newcommand{\floor}[1]{\mathsf{floor}\!\left({#1}\right)}
\newcommand{\trunc}[1]{\mathsf{trunc}\!\left({#1}\right)}
\newcommand{\ceiling}[1]{\mathsf{ceiling}\left({#1}\right)}
\newcommand{\vsum}[2]{\smashoperator[r]{\sum_{#1}^{#2}}}
\newcommand{\vxor}[2]{\smashoperator[r]{\bigoplus_{#1}^{#2}}}
\newcommand{\xor}{\oplus}
\newcommand{\band}{\binampersand}
\newcommand{\mult}{\cdot}
\newcommand{\rightarrowR}{\buildrel{\scriptstyle\mathrm{R}}\over\rightarrow}
\newcommand{\leftarrowR}{\buildrel{\scriptstyle\mathrm{R}}\over\leftarrow}

\newcommand{\RedeemScriptHash}{\mathsf{RedeemScriptHash}}

\newcommand{\pnote}[1]{\subparagraph{Note:}{#1}}
\newenvironment{pnotes}{\introlist\subparagraph{Notes:}\begin{itemize}}{\end{itemize}}

\newcommand{\affiliation}{\hairspace$^\dagger$\;}
\newcommand{\affiliationDuke}{\hairspace$^\ddagger$\;}

\begin{document}

\title{\doctitle \\
\Large \docversion}
\author{
\Large \leadauthor, \Large \coauthora
}
\date{\today}
\maketitle

\renewcommand{\abstractname}{}
\vspace{-8ex}
\begin{abstract}
\normalsize \noindent \textbf{Abstract.}

Bitcoin Hush (\BTCH) is a new research and development cryptocoin which has
many unique features compared to existing options. It avoids all transaction
history and simply imports Unspent Transaction Output (UTXO) values for
four different blockchains onto a fifth, brand-new chain. We use the Komodo Asset Chain feature to build
a coin with delayed-Proof-of-Work \cite{dPoW}, which enjoys full Bitcoin
hashpower security via notarization.

Additionally, the need for the latest two way replay protection (2WRP) algorithms are
completely avoided, since no transaction hashes are leaked
onto the new chain. This also completely avoids the problem that many Bitcoin
forks have where they inherit a very large existing chain and must sync gigabytes
of unused data.

We hope these techniques are utilized in future Bitcoin and related forks
to avoid large inefficiencies as well as potential replay attacks.

The recently released \HushList \cite{HushList} protocol is compatible with \BTCH, 
Komodo \KMD and all \KMD asset chains, which all contain \zkSNARK \cite{BCTV2015} technology.
Additionaly, \HushList is known to be compatible with
\HUSH, Zcash \ZEC, VoteCoin \VOT, Zen \ZEN, Zerocoin \ZER, Zclassic \ZCL
and the upcoming Zgold \ZAU by radix42.

\vspace{1.5ex}
\noindent This specification defines how the \BTCH cryptocoin works and how
how it builds on the foundation of \cite{Komodo},  \cite{Zcash} and \cite{Bitcoin}.

\vspace{2.5ex}
\noindent \textbf{Keywords:}~ \StrSubstitute[0]{\keywords}{,}{, }.
\end{abstract}

\vspace{-10ex}
\phantomsection
\addcontentsline{toc}{section}{\Large\nstrut{Contents}}

\renewcommand{\contentsname}{}
% http://tex.stackexchange.com/a/182744/78411
\renewcommand{\baselinestretch}{0.85}\normalsize
\tableofcontents
\renewcommand{\baselinestretch}{1.0}\normalsize
\newpage

\nsection{Introduction}

Bitcoin Hush is a "mergedrop", i.e. it is an airdrop of value from four different chains, merged together, on a new chain.

It's become common to fork the Bitcoin or Zcash network while inheriting all transaction history, which lead to a cryptocoin that has very little hashpower to protect massive amounts of data. We refer to this as a "worst of all worlds" solution and the ideas in this paper provide working examples of
avoiding this situation.

For example, Bitcoin Gold inherited about 120GB of Bitcoin history and every Bitcoin Gold full node
must download that locally, before ever getting to any Bitcoin Gold history.  The later in time
a Bitcoin fork occurs and uses the method, the larger the dataset is (currently 150GB and growing faster) and the more worse the "worst of all worlds" becomes. Just about every Bitcoin fork seen
today uses this method.

We hope that various organizations and companies realize that the Komodo Platform provides
"easy onramps" to making a new blockchain which improves on older "techniques" which amount
to forcing full node operators to use excessive amount of bandwidth and disk space for no good reason.

Additionally, the new \HushList protocol is compatible with Bitcoin Hush and provides the world
yet another place to have secure and private communication.

\nsection{What is a UTXO, really?}

Addresses spending money are called "inputs" and those receiving money in a transaction are called 
transaction outputs, TXOs. Transaction outputs that are not spent are called Unspent Transaction Output, or UTXOs.
The concept of "spentness" is a function of block height, i.e. the fact of whether a particular 
TXO is spent or not changes over time. This is why a snapshot block is needed, to define a point
in time to find unspent outputs.

A UTXO is just a blob of binary data, and has many different formats that have evolved since Bitcoin began
in 2009. The data usually contains some kind of public key or a hash of one, and other metadata. Most UTXO are around 20 bytes but multisig UTXOs can be hundreds of bytes.

For example, one of the most common forms of UTXO looks like this

	\textbf{76a914d3730f005e16bbf741ccbf60a5f66b7be930cb7a88ac}

\nsection{Hush UTXOs}

NOTE: This pre-release refers to stats before the actual snapshot, these numbers will change slighlty with the official snapshot.

This data corresponds to Hush Block Height \textbf{245496} at Jan 24, 2018 7:03:34 AM
 UTC and was extracted directly from the internal
LevelDB database using both \cite{BitcoinTools} and \cite{utxodump}.

Since Hush is a fork of the Zcash codebase, and Zcash forked from Bitcoin 0.11.2, Hush and all Zcash code
forks, to our knowledge, use the older v0.8-v0.14 LevelDB format.

At this height, \textbf{607134} UTXOs exist, in \textbf{95933} transactions and 
\textbf{3177032.96851848 HUSH}. This data is
extracted from the response of the \textbf{gettxoutsetinfo} RPC command while the full node is paused at the correct block.
Pausing the Hush daemon to allow RPC calls is not a standard function, the \textbf{pause} branch of MyHush/hush.git includes a way to achieve this. This functionality will be made a proper
command-line argument to make it easier to analyze UTXOs in the future. 

\nsection{Bitcoin UTXOs}

NOTE: This pre-release refers to stats before the actual snapshot, these numbers will change slighlty with the official snapshot.

This data corresponds to Block Height \textbf{505157} at Jan 24th, 2018 10:22:15 UTC and was
extracted directly from the internal v0.15 LevelDB database with the \cite{utxodump} tool. It took approximately 22 minutes to dump all \textbf{63.6} million UTXOs at this height.

There is \textbf{16814298.58608387 BTC} in circulation stored across \textbf{63646112} UTXOs in 
\textbf{27834789} unique transparent addresses (taddrs).
Somewhat surprisingly, there are \textbf{5605} UTXOs with exactly \textbf{0 BTC} in them and another \textbf{872314} UTXOs of a single satoshi. The latter are mostly likely from transaction spam attacks from other networks.

%Note that UTXOs of any size are taken into account, but the final sum of all UTXOs in one taddr must be above the "dust level"
%to be part of the airdrop. That is, the level at which the cost to airdrop is greater than the value of the UTXO. The level for each chain has yet to be decided.

\nsection{SUPERNET UTXOs}

To Be Determined

\nsection{DEX UTXOs}

To Be Determined

\nsection{Transporting Money to the BTCH Chain}

The \zsendmany RPC is used to efficiently send money to all the appropriate addresses, with the appropriate amount, on the new BTCH chain.

Once the final snapshot balances are known, the transparent addresses from other
networks are transformed into Komodo-compatible addresses. 
This is done by taking the RMD160 and then changing the prefix to the KMD type, then  base58\_check encoding to produce new transparent addresses for the BTCH network.

Then, many many \zsendmany transactions are performed, each with many recipients (such as 100 or 128).
Once the \zsendmany transactions for a particular chain are sent, that completes the airdrop process. Now
users can claim their airdrop via their private key.

\nsection{How To Access Your BTCH Airdrop}

The very high level idea is that the private key to your transparent address gives you access
to new funds on a new network. Extreme paranoia about private keys is not unreasonable. Fear of software bugs, malware, or anything stealing your
original \HUSH can simply be countered by moving your \HUSH to a new address,
which can be a taddr or zaddr, after the snapshot.

If you move your \HUSH post-snapshot, the privkey in question is no longer valid for your
\HUSH, and you can worry a bit less. This also protects people that might have DNS or BGP
attacks redirect them to illegitamate downloads.

For simplicity, let us assume $t_A$ is a taddr with private key $k_A$. The
\textbf{dumpprivkey} RPC method exists in all Bitcoin forks, as well as Zcash
forks, and dumps the private key in "Wallet Import Format" (WIF). This format
is accepted by the Agama Wallet. Once imported to the Agama Wallet, \BTCH
can be used on BarterDEX \cite{BarterDEX} as well.

\nsection{BTCH Chain Size}

One of the huge features of using a KMD asset chain with an airdrop is 
avoiding all transaction history, i.e. all transaction data is ignored,
only final balances in a transparent address are transported to the
new chain. When airdroping a Bitcoin fork, this currently will save
close to 150GB in chain size!

The exact numbers for the new BTCH chain will not be known until the snapshot
is complete, but current estimates are that the initial BTCH chain will be
under 0.5GB in size! 

\nsection{Running a BTCH Full Node}

Running a full \BTCH node requires downloading and syncing the \KMD chain,
which \BTCH uses for dPoW. This will use gigabytes of bandwidth at first,
and could take a day or more to fully sync depending on download speeds.

Running a full \BTCH node is not required to protect the security of the network,
nor to generate new money, but full nodes can earn the verification reward of 0.0001 BTCH per block plus
any transaction fees in that block.

Full instructions can be found on the main \BTCH Github Repo \cite{BitcoinHush}.

\nsection{Future Directions}

The \cite{HushNG} GUI is actively being worked on and will integrate Bitcoin Hush as the first additional
coin and will provide decentralized censorship-resistant communication,
powered by \zkSNARKs.

\nsection{Special Thanks}

Thanks to madbuda for providing servers and massive amounts of bandwidth for this fantastical project.

\nsection{References}

%...
\begingroup
\hfuzz=2pt
\renewcommand{\section}[2]{}
\renewcommand{\emph}[1]{\textit{#1}}
\printbibliography
\endgroup

\end{document}
